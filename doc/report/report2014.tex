%
% File eacl2014.tex
%
% Contact g.bouma@rug.nl yannick.parmentier@univ-orleans.fr
%
% Based on the instruction file for ACL 2013 
% which in turns was based on the instruction files for previous 
% ACL and EACL conferences

%% Based on the instruction file for EACL 2006 by Eneko Agirre and Sergi Balari
%% and that of ACL 2008 by Joakim Nivre and Noah Smith

\documentclass[11pt]{article}
\usepackage{eacl2014}
\usepackage{times}
\usepackage{url}
\usepackage{latexsym}
\usepackage[utf8]{inputenc}
\special{papersize=210mm,297mm} % to avoid having to use "-t a4" with dvips 
%\setlength\titlebox{6.5cm}  % You can expand the title box if you really have to

\title{Report for Artificial Intelligence TIN 172}

\author{Robert Hangu\\\And
  Simon Lindfors\\\And
  Santiago Munín González\\\And
  Carlos Tomé
   }

\date{\today}

\begin{document}
	\maketitle
	\begin{abstract}

		This document describes the design and implementation of Shrdlite. This 
		is a smaller version of the Shrdlu game, which is used in the 
		laboration of Artificial Intelligence TIN 172.

	\end{abstract}

	% ==========================================================================
	\section{Description}
	
	The game Shrdlu consists of a two-dimensional world containing object 
	stacks. These can be picked up and let down by a robot arm which acts 
	intelligently. 

	The objects have different shapes, sizes and colors. The robot may only 
	perform actions which obey physical laws, such as not putting a big object 
	into a small box or not stacking anything onto of a ball.

	Utterances can be given to the robot through a web interface, which tell it 
	what objects to move where. The robot arm then finds a way of rearranging 
	the objets in order to fulfill the user input.

	% ==========================================================================
	\section{Basic project}

	The purpose of the basic project is to implement a backend for the robot 
	arm. It receives the input, which is processed by the engine and 
	generates a list of actions that the robot must perform in order to acheive 
	the desired outcome.
	
	\subsection{Design}

	The top-level design is split into two main phases: interpretation and 
	planning.

	Interpretation deals with the parsed user input and delivers some goals 
	according to the input, that are contextualized to the world. For e.g. for the 
	input ``put the blue ball in the red box'', this should generate a goal 
	which matches all blue balls to all red boxes.
	
	Once the goals are generated, they are passed to the actual planner, which 
	outputs a list of basic actions.

	\subsection{Implementation}


	% ==========================================================================
	\section{Extensions}

	\subsection{Handling of quantifiers}
	\subsection{Ambiguity resolution}
	\subsection{Partial order planning}

	% ==========================================================================
	\section{Conclusions}

	\bibliographystyle{acl}
	\bibliography{report2014}

\end{document}
